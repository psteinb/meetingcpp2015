% 2D Image with indices
% Author: Peter Steinbach
\documentclass[tikz]{standalone}
%\documentclass[dvisvgm]{standalone}
%\def\pgfsysdriver{pgfsys-tex4ht.def}
\usepackage{units}
\usepackage{tikz}
\usetikzlibrary{calc,math,trees,positioning,arrows.meta,chains,shapes.geometric,shapes.arrows,%
    decorations.pathreplacing,decorations.pathmorphing,shapes,%
    matrix,shapes.symbols,fit,backgrounds}

 \pgfdeclarelayer{back}
 \pgfsetlayers{background,back,main}


\makeatletter
\makeatother

\begin{document}
\begin{tikzpicture}[
  show background rectangle, 
  background rectangle/.style={fill=black},
  color=white,
  help lines/.style={color=lightgray,line width=.2pt},
  ]

  \foreach \c in {0,...,3}
   {
     %cache line
     \tikzmath{
       integer \id;
       \id = \t*3;
     }
     \node (line_\c) [rectangle,dashed,draw=white,very thick,minimum width=4.5.cm,minimum height=1.1cm,anchor=west] at($(0,0)+(5*\c,0)$) {};

     \foreach \n in {0,...,3}
     {
       \tikzmath{
         integer \id;
         \id = (4*\c)+\n;
       }
       \node (mem_\id) [rectangle,dashed,draw=white,very thick,minimum width=1.cm,minimum height=1.cm,anchor=west] at($(line_\c.west)+(.1,0) + (1.1*\n,0)$) {};
     
     }
     
     
   }
  

\end{tikzpicture}
\end{document}
